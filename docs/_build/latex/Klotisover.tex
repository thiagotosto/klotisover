%% Generated by Sphinx.
\def\sphinxdocclass{report}
\documentclass[letterpaper,10pt,openany,oneside,portuges]{sphinxmanual}
\ifdefined\pdfpxdimen
   \let\sphinxpxdimen\pdfpxdimen\else\newdimen\sphinxpxdimen
\fi \sphinxpxdimen=.75bp\relax

\usepackage[utf8]{inputenc}
\ifdefined\DeclareUnicodeCharacter
 \ifdefined\DeclareUnicodeCharacterAsOptional\else
  \DeclareUnicodeCharacter{00A0}{\nobreakspace}
\fi\fi
\usepackage{cmap}
\usepackage[T1]{fontenc}
\usepackage{amsmath,amssymb,amstext}
\usepackage[portuguese]{babel}
\usepackage{times}
\usepackage[Sonny]{fncychap}
\usepackage[dontkeepoldnames,maxlistdepth=6]{sphinx}

\usepackage{geometry}

% Include hyperref last.
\usepackage{hyperref}
% Fix anchor placement for figures with captions.
\usepackage{hypcap}% it must be loaded after hyperref.
% Set up styles of URL: it should be placed after hyperref.
\urlstyle{same}

\addto\captionsportuges{\renewcommand{\figurename}{Fig.}}
\addto\captionsportuges{\renewcommand{\tablename}{Table}}
\addto\captionsportuges{\renewcommand{\literalblockname}{Listing}}

\addto\extrasportuges{\def\pageautorefname{page}}

\setcounter{tocdepth}{1}



\title{Klotisover}
\date{22 jun, 2017}
\release{}
\author{Thiago Tosto, Jose Nominato, Filipe Rangel, João Gabriel}
\newcommand{\sphinxlogo}{\vbox{}}
\renewcommand{\releasename}{Versão}
\makeindex

\begin{document}
\ifnum\catcode`\"=\active\shorthandoff{"}\fi
\maketitle
\sphinxtableofcontents
\phantomsection\label{\detokenize{index::doc}}



\chapter{Introdução}
\label{\detokenize{intro:introducao}}\label{\detokenize{intro:bem-vindo-a-documentacao-do-klotisover}}\label{\detokenize{intro::doc}}
\begin{DUlineblock}{0em}
\item[] Klotski é um jogo famoso de quebra-cabeça, onde o objetivo é retirar uma peça principal de dentro de um confinamento em meio a obstaculos móveis.
\end{DUlineblock}

\noindent{\hspace*{\fill}\sphinxincludegraphics[scale=0.7]{{klotski}.png}\hspace*{\fill}}

\begin{DUlineblock}{0em}
\item[] 
\item[] 
\end{DUlineblock}

\begin{DUlineblock}{0em}
\item[] O jogo tem uma certa complexidade e o desafio proposto por esse trabalho é desenvolver um software capaz de solucionar um caso particular desse jogo.
\end{DUlineblock}

\noindent{\hspace*{\fill}\sphinxincludegraphics{{klotski_particular}.png}\hspace*{\fill}}


\chapter{Algoritmo}
\label{\detokenize{algoritmo::doc}}\label{\detokenize{algoritmo:algoritmo}}

\section{Estrutura de dados}
\label{\detokenize{algoritmo:estrutura-de-dados}}
\begin{DUlineblock}{0em}
\item[] A estrutura de dados escolhida para solução do problema é uma árvore genérica.
\end{DUlineblock}

\begin{DUlineblock}{0em}
\item[] 
\end{DUlineblock}
\begin{figure}[htbp]
\centering

\noindent\sphinxincludegraphics{{arvore_generica}.jpg}
\end{figure}

\begin{DUlineblock}{0em}
\item[] Cada nó desta árvore corresponde a um estado da matriz que representa o jogo, aonde o nó raíz será o estado inicial da matriz.
\item[] A partir do estado inicial, é possivel enxergar 8 novos candidatos a novos estados. Cada estado representa um nó filho de seu nó pai.
\item[] Recursivamente, temos o mesmo raciocínio para cada nó, até que se chegue no estado desejado que é com a dama em sua saída.
\item[] É de suma importância que uma geração de um novo filho não gere um estado que já se encontre na árvore para que se evite loops eternos ou que
\end{DUlineblock}

cheguemos em uma configuração na qual já se tem o caminho.

\begin{DUlineblock}{0em}
\item[] O caminho do nó raíz até o nó folha que é o estado final buscado, representa a solução do problema.
\end{DUlineblock}
\begin{figure}[htbp]
\centering

\noindent\sphinxincludegraphics{{arvore_caminho}.jpg}
\end{figure}

\begin{DUlineblock}{0em}
\item[] O comprimento do caminho representa a quantidade de passos feitos para chegar na solução.
\end{DUlineblock}

\begin{DUlineblock}{0em}
\item[] 
\item[] 
\item[] 
\item[] 
\item[] 
\item[] 
\item[] 
\end{DUlineblock}


\section{Modelagem do Klotski}
\label{\detokenize{algoritmo:modelagem-do-klotski}}
\begin{DUlineblock}{0em}
\item[] A modelagem do Jogo foi feita através de uma matriz onde letras representam as peças diferentes e 0 os espaços em branco:
\end{DUlineblock}
\begin{figure}[htbp]
\centering

\noindent\sphinxincludegraphics{{modelagem}.png}
\end{figure}

\begin{DUlineblock}{0em}
\item[] A letra “D” representa a \sphinxstylestrong{dama} que é a peça que temos como objetivo tirar do quebra cabeça.
\end{DUlineblock}


\chapter{Pseudo Código}
\label{\detokenize{pseudo:pseudo-codigo}}\label{\detokenize{pseudo::doc}}

\section{Raíz}
\label{\detokenize{pseudo:raiz}}

\subsection{Gerar Árvore (principal)}
\label{\detokenize{pseudo:gerar-arvore-principal}}\begin{itemize}
\item {} 
Gera próximo estado  (a)

\item {} \begin{description}
\item[{8 vezes:}] \leavevmode\begin{itemize}
\item {} 
Testa se é solução (b)

\item {} \begin{description}
\item[{(Não) Testa se já existe (c)}] \leavevmode\begin{itemize}
\item {} 
(Não) Gera filho e chama recursividade

\end{itemize}

\end{description}

\item {} 
(Sim) Retorna até pai guardando estados no vetor solução

\end{itemize}

\end{description}

\end{itemize}


\section{2º nível de abstração}
\label{\detokenize{pseudo:o-nivel-de-abstracao}}

\subsection{Gerar próximo estado (a)}
\label{\detokenize{pseudo:gerar-proximo-estado-a}}\begin{itemize}
\item {} 
Achar lacuna

\item {} 
Achar periféricos

\item {} \begin{description}
\item[{Para cada periférico:}] \leavevmode\begin{itemize}
\item {} 
Tentar mover periférico e guardar estado no vetor estados

\end{itemize}

\end{description}

\item {} 
Retorna estados

\end{itemize}


\subsection{Solução teste (b)}
\label{\detokenize{pseudo:solucao-teste-b}}\begin{itemize}
\item {} 
compara matriz (i)

\end{itemize}

\begin{DUlineblock}{0em}
\item[] 
\item[] 
\end{DUlineblock}


\subsection{Testa se já existe(busca na árvore) (c)}
\label{\detokenize{pseudo:testa-se-ja-existe-busca-na-arvore-c}}\begin{quote}

Parâmetros: nó raiz; nó buscado(estado)
\begin{itemize}
\item {} \begin{description}
\item[{Para cada filho:}] \leavevmode\begin{itemize}
\item {} \begin{description}
\item[{Testa se filho é igual a nó buscado(compara matriz):}] \leavevmode\begin{itemize}
\item {} 
(sim) retorna verdadeiro

\item {} 
(não) Chama recursividade passando filho como raíz

\end{itemize}

\end{description}

\end{itemize}

\end{description}

\item {} 
Retorna falso

\end{itemize}
\end{quote}


\subsection{Gera filho}
\label{\detokenize{pseudo:gera-filho}}\begin{quote}

Parâmetros: pai; estado;
\begin{itemize}
\item {} 
«Instancia» filho.

\item {} 
Setar pai (filho.pai = pai).

\item {} 
Setar estado (filho.estado = estado).

\item {} 
Anula filhos (filhos.filhos{[}{]}).

\item {} 
Retorna filho

\end{itemize}
\end{quote}


\section{3º nível de abstração}
\label{\detokenize{pseudo:id1}}

\subsection{Compara matriz (i)}
\label{\detokenize{pseudo:compara-matriz-i}}\begin{quote}

Parâmetros: matriz 1; matriz 2;
\begin{itemize}
\item {} \begin{description}
\item[{Testa se matriz1 e matriz2 são nulas:}] \leavevmode\begin{itemize}
\item {} 
(não) compara

\item {} 
(sim) retorna falso

\end{itemize}

\end{description}

\end{itemize}
\end{quote}



\renewcommand{\indexname}{Índice}
\printindex
\end{document}